\chapter{Conclusion}
\section{Summary}
This research proposes a new model for the Contacts applications on modern smart phones. The new model improves the current Contacts applications by introducing three novel capabilities:

\begin{itemize}
    \item {Searching for contacts through their miscellaneous information.}
    \item {Efficiently retaining knowledge of contacts by using a tags system.} 
    \item {Establishing a \textit{Personal Social Network} consists of the relationships between the contacts, the network can be traversed and explored easily.}
\end{itemize}

By introducing these capabilities, the model helps its users to accomplish new tasks which are not currently handled by modern Contacts applications. It provides users a more powerful tool to manage their contacts and also easy to use. As a result, users can search for their contacts faster and recall contact information easier.

The model is implemented to become Graphy - a fully functional prototype of a Contacts application on iOS and Android. Besides normal features of a Contacts application, Graphy provides extra functionalities such as: allowing users to add customized-information tags to their contacts, automatically adding contextual-information tags like creation date and creation location to the contacts, allowing users to establish relationships between contacts and traverse the relationships network, providing users with a powerful search feature so they can search for contacts through tags, relationships or normal information. Moreover, the prototype is backed by a server which delivers a data synchronization service between multiple devices.

Two experiments were conducted in this research. The first experiment is a user study which focuses on evaluating the user acceptance of the new capabilities of the prototype. The second experiment aim at testing the efficiency of the model in synchronizing its data between multiple devices. The first experiment's results shows that participants frequently use the new features. The number of tags and relationships created by the participants are almost equal to the amount of traditional information (names, phone numbers, emails, etc.). Furthermore, they even preferred searching by tags and relationships to searching by traditional information. The results of the second experiment indicates that it is easy to exchange tags and relationships data. Our system was built on low computing power hardware but can still achieved good performance.

\section{Contributions}
This thesis has the following contributions:

\begin{enumerate}
    \item Introducing a new method to retain and look up information of contacts through a tag system. This method helps users to find the right contact with any particular pieces of information and allow them to easily store miscellaneous data of people.
    \item Presenting a novel \textit{Personal Social Network} based on relationships between contacts. On the one hand, this network share many characteristics with a normal social network since it connects information of individuals. On the other hand, it differentiates itself by the \textit{personal} nature of the content. With this network we can answer a whole new class of user queries such as ``Find all colleagues of contact X'', ``Find the spouse of contact Y''. This type of queries is currently impossible to accomplish in Contacts applications nowadays. Moreover, we believe this \textit{Personal Social Network} has many undisclosed potentials which can lead to different research directions.
    \item Building a fully functional prototype of the whole system from the mobile clients to the backend server. The prototype uses state-of-the-art technologies so it can serve as a reference for further research on the topic.
    \item Evaluating the prototype in terms of both user experience and operational efficiency. Two experiments were carried out: a user acceptance study and a system performance measurement.
\end{enumerate}

\section{Future Work}
abc


%\section{Summary and Expected Contributions}
%In this research, we proposed a new model for the contacts management applications which will achieve three goals:
%\begin{itemize}
%    \item {Introduce a new way to look up contacts by using their miscellaneous information.}
%    \item {Efficiently retain knowledge of contacts by using a tag system.} 
%    \item {Enable users to establish relationships between their contacts then traverse and explore the relationships with ease.}
%\end{itemize}
%
%By achieving these goals, our model helps its users to accomplish new tasks which are not currently supported by modern Contacts applications. It is expected to provide users a more powerful tool to manage their contacts and also easy to use. As a result, users can search for their contacts faster and recall contact information easier. Moreover, the novel ``reversed social network'' proposed by the research is expected to open a new research direction in exploring and utilizing the relationships between contacts.
%
%\section{Future Work}
%Due to the time constraint, there are several areas our research has not been able to explore, namely:
%
%\begin{itemize}
%    \item {Contact relationships suggestions: In a future extension, our prototype could be able to suggest relationships to users instead of letting them manually identify them. There are many possible directions could be used for suggestions such as examining area codes in the phone numbers, extracting information from the device's mailbox, analyzing existing mutual connections.}
%    \item {Unify with online social networks: Our ``reversed social network'' can be connected to online social networks like Facebook, LinkedIn. As a result, we can pull updated information from actual social networks into our application.} 
%    \item {Network visualization: Visualizing the ``reversed social network'' could provide users benefits like assisting them in having a better overview and insight of their circle of friends.}
%    \item {Ubiquitous computing: Our architecture can be extended to support operations on more computing devices like smart watches, desktops and the web.}
%\end{itemize}

%This study proposes a new approach to the Contacts application. By introducing new types of information in combination with capturing social data, the new Contacts application provides users a better way of managing their contacts. Furthermore, the study presents a novel internal network of relationships among contacts which helps users explore the contacts space as a whole - a task which is still impossible for the Contacts application nowadays.

%Although the experiment still has limitations, it has shown that our approach has significantly improved the Contacts application. Our future work will focus on carrying out large scale, comprehensive experiments together with exploiting more potentials of the new Contacts application such as relationships recommendation, contextual searching.
