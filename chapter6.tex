\chapter{Conclusion}
\section{Summary}
This research proposes a new model for the Contacts applications on modern smart phones. The new model improves the current Contacts applications by introducing three novel capabilities:

\begin{itemize}
    \item {Searching for contacts through their miscellaneous information.}
    \item {Efficiently retaining knowledge of contacts by using a tags system.} 
    \item {Establishing a \textit{Personal Social Network} which consists of the relationships between the contacts, the network can be traversed and explored easily.}
\end{itemize}

By introducing these capabilities, the model helps its users to accomplish new tasks which are not currently handled by modern Contacts applications. It provides users with a more powerful tool to manage their contacts and is also easy to use. As a result, users can search for their contacts faster and recall contact information easier.

The model is implemented to become Graphy - a fully functional prototype of a Contacts application on iOS and Android. Besides normal features of a Contacts application, Graphy provides extra functionalities such as: allowing users to add customized-information tags to their contacts, automatically adding contextual-information tags like creation date and creation location to the contacts, allowing users to establish relationships between contacts and traverse the relationships network, providing users with a powerful search feature so they can search for contacts through tags, relationships or normal information. Moreover, the prototype is backed by a server which delivers a data synchronization service between multiple devices.

Two experiments were conducted in this research. The first experiment is a user study which focuses on evaluating the user acceptance of the new capabilities of the prototype. The second experiment aims at testing the efficiency of the model in synchronizing its data between multiple devices. The first experiment's results show that participants frequently use the new features. The number of tags and relationships created by the participants are almost equal to the amount of traditional information (names, phone numbers, emails, etc.). Furthermore, they even preferred searching by tags and relationships to searching by traditional information. The results of the second experiment indicates that it is easy to exchange tags and relationships data. Our system was built on low computing power hardware but could still achieve good performance.

\section{Contributions}
This thesis has the following contributions:

\begin{enumerate}
    \item Introducing a new method to retain and look up information of contacts through a tag system. This method helps users to find the right contact with any particular pieces of information and allow them to easily store miscellaneous data of people.
    \item Presenting a novel \textit{Personal Social Network} based on relationships between contacts. On the one hand, this network share many characteristics with a normal social network because it connects various information of individuals. On the other hand, it differentiates itself by the \textit{personal} nature of the content. With this network we can answer a whole new class of user queries such as ``Find all colleagues of contact X'', ``Find the spouse of contact Y''. This type of queries is currently impossible to solve in Contacts applications nowadays. Moreover, we believe that this \textit{Personal Social Network} has many unexplored potentials which can lead to different research directions.
    \item Building a fully functional prototype of the whole system from the mobile clients to the backend server. The prototype uses state-of-the-art technologies so it can serve as a reference for further research on the topic.
    \item Evaluating the prototype in terms of both user experience and operational efficiency. Two experiments were carried out: a user acceptance study and a system performance measurement.
\end{enumerate}

\section{Future Work}
Due to the time constraint, there are several areas our research has not been able to explore, namely:

\begin{itemize}
    \item {\textbf{Unifying with online social networks}: It is remarkably beneficial if our \textit{Personal Social Network} can connect to online social networks like Facebook or LinkedIn. Subsequently, we can pull the latest information from those social networks into our system to keep the contacts up to date. For instance, when a contact moves to another city and updates his address on Facebook, Graphy can detect this change and automatically updates its local database.}
    \item {\textbf{Relationships recommendation}: We can investigate the area of Recommendation System to integrate it into Graphy. As a result, our system will be able to suggest relationships to users instead of letting them manually identify the relationships. Many possible directions could be used for recommendation such as analyzing existing mutual connections, examining area codes in the phone numbers, extracting information from the device's mailbox, etc. This improvement will help the users extend their \textit{Personal Social Network} much easier.} 
    \item {\textbf{Advanced auto-added tags}: Currently we only implement 2 auto-added tags, i.e. Created Date and Created Location. These tags help the users to remember when and where they created their contacts. In the future, we can carry out more research on contextual information to introduce more auto-added tags. For example, Graphy can access the calendar of the smart phones, hence it knows all the events or meetings which happen when the users create a contact. These pieces of information are definitely valuable for finding that contact at a later time.}
    \item {\textbf{Network visualization}: At the moment, users can freely traverse the \textit{Personal Social Network} by navigating from one contact to another via the relationship between them. However, it is still impossible for them to have an overview of the whole network. Therefore, we believe that visualizing the \textit{Personal Social Network} is essential. It can assist users to have a better overview and insights into their circle of friends.}
    \item {\textbf{A larger scale user study}: The number of people participating in our experiment is still limited. In the future, we will conduct another study with a greater number of participants in a longer period of time. Furthermore, we will also improve the logging method and questionnaires to have a better understanding of the users.}

%    \item {Ubiquitous computing: Our architecture can be extended to support operations on more computing devices like smart watches, desktops and the web.}
\end{itemize}

%\section{Summary and Expected Contributions}
%In this research, we proposed a new model for the contacts management applications which will achieve three goals:
%\begin{itemize}
%    \item {Introduce a new way to look up contacts by using their miscellaneous information.}
%    \item {Efficiently retain knowledge of contacts by using a tag system.} 
%    \item {Enable users to establish relationships between their contacts then traverse and explore the relationships with ease.}
%\end{itemize}
%
%By achieving these goals, our model helps its users to accomplish new tasks which are not currently supported by modern Contacts applications. It is expected to provide users a more powerful tool to manage their contacts and also easy to use. As a result, users can search for their contacts faster and recall contact information easier. Moreover, the novel ``reversed social network'' proposed by the research is expected to open a new research direction in exploring and utilizing the relationships between contacts.
%
%\section{Future Work}


%This study proposes a new approach to the Contacts application. By introducing new types of information in combination with capturing social data, the new Contacts application provides users a better way of managing their contacts. Furthermore, the study presents a novel internal network of relationships among contacts which helps users explore the contacts space as a whole - a task which is still impossible for the Contacts application nowadays.

%Although the experiment still has limitations, it has shown that our approach has significantly improved the Contacts application. Our future work will focus on carrying out large scale, comprehensive experiments together with exploiting more potentials of the new Contacts application such as relationships recommendation, contextual searching.
