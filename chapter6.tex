\chapter{Conclusion}
\section{Summary and Expected Contributions}
In this research, we proposed a new model for the contacts management applications which will achieve three goals:
\begin{itemize}
    \item {Introduce a new way to look up contacts by using their miscellaneous information.}
    \item {Efficiently retain knowledge of contacts by using a tag system.} 
    \item {Enable users to establish relationships between their contacts then traverse and explore the relationships with ease.}
\end{itemize}

By achieving these goals, our model helps its users to accomplish new tasks which are not currently supported by modern Contacts applications. It is expected to provide users a more powerful tool to manage their contacts and also easy to use. As a result, users can search for their contacts faster and recall contact information easier. Moreover, the novel ``reversed social network'' proposed by the research is expected to open a new research direction in exploring and utilizing the relationships between contacts.

\section{Timeline}
The expected timeline for this research is as follows:

\begin{itemize}
    \item {Complete implementation, start distributing the prototype to users: June 30}
    \item {Collect users' feedback, start analyzing users' data: July 15} 
    \item {Finish writing thesis, ready for the defense: July 31}
\end{itemize}

\section{Future Work}
Due to the time constraint, there are several areas our research has not been able to explore, namely:

\begin{itemize}
    \item {Contact relationships suggestions: In a future extension, our prototype could be able to suggest relationships to users instead of letting them manually identify them. There are many possible directions could be used for suggestions such as examining area codes in the phone numbers, extracting information from the device's mailbox, analyzing existing mutual connections.}
    \item {Unify with online social networks: Our ``reversed social network'' can be connected to online social networks like Facebook, LinkedIn. As a result, we can pull updated information from actual social networks into our application.} 
    \item {Network visualization: Visualizing the ``reversed social network'' could provide users benefits like assisting them in having a better overview and insight of their circle of friends.}
    \item {Ubiquitous computing: Our architecture can be extended to support operations on more computing devices like smart watches, desktops and the web.}
\end{itemize}

%This study proposes a new approach to the Contacts application. By introducing new types of information in combination with capturing social data, the new Contacts application provides users a better way of managing their contacts. Furthermore, the study presents a novel internal network of relationships among contacts which helps users explore the contacts space as a whole - a task which is still impossible for the Contacts application nowadays.

%Although the experiment still has limitations, it has shown that our approach has significantly improved the Contacts application. Our future work will focus on carrying out large scale, comprehensive experiments together with exploiting more potentials of the new Contacts application such as relationships recommendation, contextual searching.
