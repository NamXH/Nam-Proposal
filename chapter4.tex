\chapter{Experiment and Evaluation - User Study}
In this chapter, the Graphy prototype is evaluated in a user study. The objective of the study is to evaluate the user acceptance of Graphy to see whether it achieves its three goals: allowing users to find contacts by their miscellaneous information, efficiently retain knowledge of contacts, and enabling users to establish relationships between contacts then traverse the relationships with ease. The chapter contains three sections describing the hypotheses of the study, the detailed experimental setup, and the results.
\section{Hypotheses}
The main goal of this user study is to evaluate Graphy's functionalities in real use over a period of time. The evaluation mainly aims at testing the following hypotheses:
\begin{enumerate}
  \item Graphy allows users to find contacts by their miscellaneous information. Users often use this functionality.
  \item Graphy allows users to efficiently retain knowledge of contacts. Users utilize this feature to maintain a considerable amount of information, especially miscellaneous information.
  \item Graphy allows users to establish relationships between contacts then traverse the relationship network with ease. Users create a lot of relationships and actively traverse the relationships during their usage.
\end{enumerate}
\section{Experimental Setup}
The method of this user study is to recruit people to download the Graphy application to their smart phones after signing a consent form to participate in the study. The participants then use it in an uncontrolled environment for two weeks. The Graphy application records users' activities like creating, editing, and searching for contacts into a list of numbers on a summary screen which is part of the user interface. After using it for two weeks, the users go to the summary screen and copy all the numbers on it to an online form. The summary screen and the online form are illustrated in figures \ref{fig:summary_page}) and \ref{fig:survey_form}).

\begin{figure}[!h]
\begin{centering}
\includegraphics[scale=0.33]{pics/summary_page.png}
\caption{Graphy's Summary Page}\label{fig:summary_page}
\end{centering}
\end{figure}

\begin{figure}[!h]
\begin{centering}
\includegraphics[scale=0.6]{pics/survey_form.png}
\caption{Graphy's Online Form}\label{fig:survey_form}
\end{centering}
\end{figure}

To recruit participants, we sent email invitations to our contacts list and also advertised the study on Facebook. As a result, twenty one people installed the application and finished the online form. The form is divided into two categories: one for tags (which represent pieces of miscellaneous information), one for relationships. Each category has two sections: the contact data and the search pattern. The contact data section accumulates information in the users' contacts list. For example, it records the total number of contacts, the total number of tags and relationships created during the study. This section allows us to examine the way participants store their information using Graphy, how they utilize the newly introduced elements in our application. The search pattern section collects data when a participant uses the search function or traverse the relationships network. For example, it tracks the frequency a user searches for contacts using tags, the number of tags used in each search, and how many times the user navigates from one contact to another through the relationship network. This section helps us to understand how people consume the data they store in Graphy: whether they use the newly introduced elements to search or they still use traditional information like first name and last name, whether they use the relationships network to look for a contact, etc.
\section{Results and Discussions}
This section presents the results of the study. The section contains two parts. The first part \textit{\nameref{results_tags_relationships}} \ref{results_tags_relationships} looks into the information participants store in Graphy which includes tags, relationships, and traditional information like names and phone numbers. The second part \textit{\nameref{results_search}} \ref{results_search} examines how participants consume the data they store in Graphy through searching and navigating the relationships network.
\subsection{Tags and Relationships}\label{results_tags_relationships}
The first few metrics of Graphy aim at evaluating how intensively the participants use the system. More specifically, the first metric that we calculate is Active Contacts Total - the total number of contacts which were viewed, edited or added during the study. All data from the twenty one respondents shows a moderately high number of Active Contacts Total. On average, a person uses approximately 33 contacts in 2 weeks, which means well over 2 contacts per day. The vast majority of the users have more than 15 active contacts as opposed to only a modest 14\% of them have less than 15. The Active Contacts Total data can be seen in Table \ref{table:active_contacts} and the group of Figures \ref{fig:active_contacts}, \ref{fig:tags_total}, \ref{fig:relationships_total}. From the data, we can infer that participants use Graphy quite a lot. 

\begin{table}[htbp]
  \centering
  \caption{Total Numbers of Active Contacts, Tags, and Relationships}\label{table:active_contacts} 
    \begin{tabular}{rrrrr}
    \toprule
    Active Contacts Total & Tags Total & Tags Average & Relationships Total & Relationships Average \\
    \midrule
    46    & 180   & 3.913 & 94    & 2.0435 \\
    72    & 241   & 3.3472 & 146   & 2.0278 \\
    55    & 127   & 2.3091 & 67    & 1.2182 \\
    16    & 27    & 1.6875 & 30    & 1.875 \\
    50    & 101   & 2.02  & 87    & 1.74 \\
    9     & 22    & 2.4444 & 7     & 0.7778 \\
    8     & 29    & 3.625 & 12    & 1.5 \\
    49    & 165   & 3.3673 & 86    & 1.7551 \\
    20    & 48    & 2.4   & 39    & 1.95 \\
    18    & 69    & 3.8333 & 34    & 1.8889 \\
    32    & 68    & 2.125 & 45    & 1.4063 \\
    29    & 67    & 2.3103 & 33    & 1.1379 \\
    23    & 60    & 2.6087 & 29    & 1.2609 \\
    24    & 47    & 1.9583 & 38    & 1.5833 \\
    30    & 102   & 3.4   & 58    & 1.9333 \\
    63    & 207   & 3.2857 & 128   & 2.0317 \\
    8     & 26    & 3.25  & 10    & 1.25 \\
    52    & 111   & 2.1346 & 80    & 1.5385 \\
    41    & 85    & 2.0732 & 71    & 1.7317 \\
    22    & 41    & 1.8636 & 41    & 1.8636 \\
    17    & 56    & 3.2941 & 31    & 1.8235 \\
          &       &       &       &  \\
          \multicolumn{5}{l}{\textit{Average:}} \\ 
    \textbf{32.5714} &  \textbf{89.4762} &  \textbf{2.7262} &  \textbf{55.5238} &  \textbf{1.6351} \\
    \bottomrule
    \end{tabular}%
  \label{tab:addlabel}%
\end{table}%

Table \ref{table:active_contacts} also shows the other two metrics: Tags Total and Relationships Total which are the total number of tags and relationships respectively. Regarding tags, the average total number is almost 90. A significant proportion of participants, 38\%, maintain more than 100 tags. Nobody has less than 20 tags and 2 people strikingly have more than 200 tags. Dividing the number of tags by the number of active contacts gives us the average of tags per contact which we call Tag Average. To be more specific, the Tag Average among all participants is roughly 2.73 tags per contact. The highest Tag Average is 3.91 and the lowest is 1.68. These are promising numbers. Generally, every user creates at least one tag for his/her contacts and on average almost 3 tags per contact. Because tags represent miscellaneous information users put into their contacts, the higher the numbers, the more miscellaneous information is created. From the collected data here, we can conclude that participants need about 3 pieces of miscellaneous information for every contact. Notably, the tags recorded here do not include Auto Added Tag (tags that the system automatically adds to a contact like \textit{Date Created}, \textit{Location Created}). In comparison with tags, the numbers of relationships are slightly smaller. On average, a respondent creates 55.52 relationships over the course of 14 days. Only 2 users, that is 9.5\%, maintain more than 100 relationships. When it comes to Relationship Average, typically each contact has 1.63 relationships. There are three people maintaining the highest Relationship Averages at 2.02, 2.03, and 2.04. On the contrary, there is one person who has only 0.78 relationships per contact. Although the users tend to create fewer relationships than tags, 1.63 relationships per contact is still a very positive indicator.

\begin{figure}[!htbp]
\begin{centering}
\includegraphics[scale=0.5]{pics/active_contacts.png}
\caption{Number of Active Contacts}\label{fig:active_contacts}
\includegraphics[scale=0.5]{pics/tags_total.png}
\caption{Number of Tags}\label{fig:tags_total}
\includegraphics[scale=0.5]{pics/relationships_total.png}
\caption{Number of Relationships}\label{fig:relationships_total}
\end{centering}
\end{figure}

Apart from calculating the total numbers of tags and relationships, we also recorded tag and relationship types. Table \ref{table:tag_relationship_types} shows how many types of tags and relationships were created during the experiment. The Average Tag Type Count is about 65. The person with the most tag types has 126 types while the person who creates the least types has 27. Another thing which stands out in this table is that there are two people who have more tag types than tags. They both have very small numbers of tag types which are 29 and 30. These unusual situations can be explained by two reasons. First, Graphy has 4 pre-defined tags which are hard-coded in the application: \textit{Colleague}, \textit{Important}, \textit{Created Date}, \textit{Created Location}. We create these pre-defined tags as examples to guide users in how to use tags. These 4 tags are added to the number of tag types so it is possible that the users get 4 extra types without actually using them. Second, every time a user creates a tag type, the type is stored in the database so the user can use it for multiple contacts. If the user creates some types then never uses them, or uses the types on a contact then deletes that contact, the total number of tag types will be increased by the amount of unused types. This fact reveals a limitation in our system design. Ideally, Graphy should have a clean up function which deletes unused types after a long period of time. To some extent, the ratio of the number of tags to the number of tag types implies how many times a tag is reused. From table \ref{table:tag_relationship_types}, the average ratio is roughly 1.3 tags for each tag type. Noticeably, there is one participant who maintains a ratio of exactly 1. Overall, this \textit{reused} ratio is fairly low. On the one hand, that means miscellaneous information is broad and needs many different tag types to cover. On the other hand, it exposes another limitation of the system: poor recommendation for tag/relationship types. At the moment, our application lets people enter new tag type names or pick an existing tag type from a list. A better solution should be giving suggestions based on the existing types as users enter type names. Moreover, Graphy is currently case-sensitive which in turn may be inflexible. For instance, it will consider these two names different types ``Important'' and ``important''. Therefore, making Graphy case-insensitive or equipping it with the capability of detecting similar words is quite important. These two improvements will be reserved for future work. In comparison to tag types, the number of relationship types is smaller by a large margin. The average number of relationship types is only 17, nearly 4 times smaller than the tag types. Concerning the relationships reused ratio, it is just almost three times as big as the ratio of tags. Notably, there is one respondent having the relationship reused ratio of 7.9 which is substantially high while the lowest relationship reused ratio is practically 2 relationships for each relationship type. These numbers suggest that relationship types are very well used, and people appear to have fewer kinds of relationship.

\begin{table}[!htbp]
  \centering
  \caption{Tag and Relationship Types}\label{table:tag_relationship_types} 
    \begin{tabular}{rrrrrr}
    \toprule
    Tag Total & \specialcell[t]{Tag:\\Type Count} & \specialcell[t]{Tag:\\Total/Type} & Relationship Total & \specialcell[t]{Relationship:\\Type Count} & \specialcell[t]{Relationship:\\Total/Type} \\
    \midrule
    180   & 114   & 1.5789 & 94    & 23    & 4.087 \\
    241   & 106   & 2.2736 & 146   & 36    & 4.0556 \\
    127   & 126   & 1.0079 & 67    & 36    & 1.8611 \\
    27    & 27    & 1     & 30    & 14    & 2.1429 \\
    101   & 91    & 1.1099 & 87    & 11    & 7.9091 \\
    22    & 29    & 0.7586 & 7     & 4     & 1.75 \\
    29    & 30    & 0.9667 & 12    & 5     & 2.4 \\
    165   & 117   & 1.4103 & 86    & 27    & 3.1852 \\
    48    & 39    & 1.2308 & 39    & 15    & 2.6 \\
    69    & 52    & 1.3269 & 34    & 10    & 3.4 \\
    68    & 67    & 1.0149 & 45    & 23    & 1.9565 \\
    67    & 64    & 1.0469 & 33    & 18    & 1.8333 \\
    60    & 42    & 1.4286 & 29    & 15    & 1.9333 \\
    47    & 39    & 1.2051 & 38    & 17    & 2.2353 \\
    102   & 56    & 1.8214 & 58    & 16    & 3.625 \\
    207   & 94    & 2.2021 & 128   & 33    & 3.8788 \\
    26    & 20    & 1.3   & 10    & 5     & 2 \\
    111   & 104   & 1.0673 & 80    & 20    & 4 \\
    85    & 78    & 1.0897 & 71    & 10    & 7.1 \\
    41    & 39    & 1.0513 & 41    & 13    & 3.1538 \\
    56    & 45    & 1.2444 & 31    & 10    & 3.1 \\
          &       &       &       &       &  \\
          \multicolumn{6}{l}{\textit{Average:}} \\ 
    \textbf{89.4762} & \textbf{65.6667} & \textbf{1.2922} & \textbf{55.5238} & \textbf{17.1905} & \textbf{3.2479} \\
    \bottomrule
    \end{tabular}%
  \label{tab:addlabel}%
\end{table}%

When it comes to measuring the impact of Graphy's new elements to a contact, we use another two metrics: Tag Weight Average and Relationship Weight Average. The way these metrics are calculated is as follows: on a contact we count the number of tag fields and the number of relationship fields then divide them by the number of all fields from which we get Tag Weight and Relationship Weight. The averages of Tag and Relationship Weight are then computed among all the contacts of a participant. The histograms of the two metrics are illustrated by Figure \ref{fig:tag_weight_histogram} and \ref{fig:relationship_weight_histogram} and the detailed results can be found in Appendix A. As regards Tag Weight Average, each participant generally maintains a considerable high number at roughly 27\%. Remarkably, the vast majority of participants (10 people) have Tag Weight Average between 31\% and 35\%. Nobody has less than 16\% or more than 40\% on this metric. It is worth noting that Auto Added Tags such as \textit{Date Created} and \textit{Location Created} are not included in the calculation. If we add these two simple tags into the formula, Tag Weight Average can be increased by a large margin. By comparison, Relationship Weight is a little lower than Tag Weight at approximately 17\%. Only one respondent has Relationship Weight Average by less than 10\% while the greatest proportion of respondents, 57\% (12 people), keep this metric in the 16-20\% range. There are four users with Tag Weight Average between 11\% and 15\%, at the same time another four users have the metric between 21\% and 25\%. These data are again very promising. On average, tags and relationships together serve almost half the information in a contact. The pie chart \ref{fig:average_contact_fields} displays more clearly that Graphy's newly introduced elements (tags and relationships) are practically as many as traditional pieces of information (e.g. names, phone numbers, emails, etc.). From our experience, a contact typically contains 3 to 5 traditional fields: a first name and last name, one or two phone numbers, an email, and sometimes the organization that the contact belongs to (interestingly, the organization is often placed together with the names). Since traditional information takes up 56\% of the total fields, the number of tag fields is between 2 and 3 while the number of relationship fields is 1 or 2. This observation perfectly matches with the Tag Average and Relationship Average data we have just discussed in the previous paragraph (which are 2.7 and 1.6 respectively). To conclude, the results show that tags and relationships constitute a very large part of a contact. They contain special pieces of information which is important to retain knowledge of users' contacts.

\begin{figure}[!h]
\begin{centering}
\includegraphics[scale=0.6]{pics/tag_weight_histogram.png}
\caption{Average Tag Weight Distribution among Participants}\label{fig:tag_weight_histogram}
\end{centering}
\end{figure}

\begin{figure}[!h]
\begin{centering}
\includegraphics[scale=0.6]{pics/relationship_weight_histogram.png}
\caption{Average Relationship Weight Distribution among Participants}\label{fig:relationship_weight_histogram}
\end{centering}
\end{figure}

\begin{figure}[!h]
\begin{centering}
\includegraphics[scale=0.7]{pics/average_contact_fields.png}
\caption{Average Proportion of Fields in A Contact}\label{fig:average_contact_fields}
\end{centering}
\end{figure}

\subsection{Contacts Search and Relationship Traversal}\label{results_search}

The group of metrics we analyze in the previous section focuses on investigating how people store their contacts information in Graphy. In this section, we are going to explore the other group of metrics which examines the way people consume their stored information.

\begin{figure}[!h]
\begin{centering}
\includegraphics[scale=0.7]{pics/search_total.png}
\caption{Total Numbers of Searches}\label{fig:search_total}
\end{centering}
\end{figure}

The first aspect to go into is how users search for contacts. In order to investigate this, we record the number of search activities they perform during the experiment. It is important to note that we only count successful searches which yield at least one result. On average, a user looks up just over 32 contacts over the course of 14 days which equals to more than a search per day. This means the users use Graphy to retrieve information quite often. Particularly, one participant uses the search function 73 times while the two least active users use it 12 times. The detailed results of the study can be found in \autoref{fig:search_total}. Among all the search activities, the first type of search we want to mention is the Traditional Search. A Traditional Search occurs when a user looks up contacts using \textit{traditional information} like names, phone numbers, emails, organization, etc. This is the type of search people are currently performing in contacts management applications nowadays. The average of Traditional Searches is 8.67 times or approximately 27\% of the total number of searches. The highest number of Traditional Searches is 19 (61\%) while there are two users who do not use traditional information to search at all. The data here is surprisingly low. It seems participants are moving away from the traditional method of searching and utilize the newly introduced elements in Graphy to search. Regarding tags, the number of Tag Searches are strikingly high. Although tags take up only 27\% information of a contact (\autoref{fig:average_contact_fields}), they are the pieces of information participants use the most for searching. In general, a person uses tags to search more than 58\% of the time. On the contrary, a modest 14.6\% of the total searches is done via relationships. The average proportions of search types are illustrated clearly by \autoref{fig:search_percentage}. Another outstanding point in the data is that one participant use Tag Searches 13 times in his/her total 14 searches but only use Traditional Search once and never use Relationship Search. There are actually two people who do not use Relationship Search in our study while the lowest Tag Search percentage is over 36\%. In conclusion, the vast majority of searches are accomplished through tags which are above two times as many as Traditional Searches while Relationship Searches are the lowest among the 3 types.

\begin{figure}[!h]
\begin{centering}
\includegraphics[scale=0.68]{pics/search_percentage.png}
\caption{Average Proportion of Searches}\label{fig:search_percentage}
\end{centering}
\end{figure}

The second aspect we investigate in the experiment is the way people explore the relationships network. In fact, traversing the relationships network can be considered a special form of searching for information. Every time a user travels from one contact to another through the relationship connection between them we increase the tracking variable \textit{Relationship Navigations} by one. By contrast to the low number of Relationship Searches, the average of Relationship Navigations is significantly high: at 19, almost two thirds of the total number of searches. In other words, each participant uses this feature nearly 1.5 times everyday. Especially, one participant navigates his/her relationships network 62 times during the study. According to \autoref{fig:relationship_navigation_histogram}, the greatest proportion of users, that is 9 (43\%), traverse the network between 11 and 20 times, and only a minority of them (23\%) uses the functionality less than 10 times. From the data, we can deduce that relationships are not often used directly for searching. Instead, they turn out to be valuable when the users want to find a contact through the information of another contact. This fact is actually understandable. Directly searching for a relationship can only answer a special type of requests, for example ``Find me all contacts who have a daughter!''. This type of requests is definitely rare in real life scenarios so direct relationships searches are few. On the other hand, relationships navigations can help users to find a contact when the only thing they remember is that contact is a friend/colleague/relative of someone. This type of request is surely more common. Moreover, navigating the relationships can also be used to explore the whole circle of friends. Therefore, it is frequently used.

\begin{figure}[!h]
\begin{centering}
\includegraphics[scale=0.68]{pics/relationship_navigation_histogram.png}
\caption{Relationship Navigations Distribution among Participants}\label{fig:relationship_navigation_histogram}
\end{centering}
\end{figure}

The last aspect of the contact searches we examine is the number of tags and relationships used in each search. Regarding tags, the average number of tags used in all the searches is just over 19. Since the average total number of tag searches is 18.7, the average number of tags per search is practically 1. It means the users often use only 1 tag for every search. Only a minority of participants, 23\%, has Tags Per Search higher than 1. In a similar fashion, participants use exactly 1 relationship in each relationship search. This result reveals that people tend to be \textit{lazy} while searching. They only enter one criterion then pursue the search. Perhaps they like scrolling the list of search results to find the desired contact more than typing more search criteria to filter the list further. It is totally reasonable because we all know people do not like typing a lot. Another possibility is that users usually can only recall the most distinct characteristic of the contact they want to find. These conclusions lead us to an improvement we can add to Graphy in a future version: recommending relevant tags while users type search criteria.

To sum up, we have just investigated the way participants search for contacts and navigate their relationships networks. The results are very promising. People often use the newly introduced elements in Graphy to look for information. Despite taking up 56\% of a contact, traditional information like names or organizations only contribute a mere 27\% of the total search while tags outstrip it and account for 58\% of the searches. As for relationships, although they are rarely used in direct searches, users utilize them to explore the relationships connections and to find a contact when it is connected with someone more memorable. However, the data can be a little in favor of tags and relationships because participants feel that they are using a special application with exclusive features so they try the features, instead of actually using them. But even so, we still think tags and relationships have confirmed their values. In the future, we will conduct another study which has a timeline to see if people increasingly use tags and relationships instead of traditional information over time.

\section{Summary}\label{results_summary}
According to the analyses and discussions above, we can say that the user study has confirmed our 3 hypotheses. As regards the first hypothesis, Graphy is not only capable of finding contacts by miscellaneous information, but it also does it very well. Searching by tags ranks first among all searches with a striking percentage at 58\% while traditional searches take up only 27\%. It is also worth noting that to the best of our knowledge Graphy is the only application that can perform this type of searching while popular contacts management applications on iOS or Android can only look up a set of pre-defined fields. As for the second hypothesis, our system clearly allows people to retain knowledge of their contacts through the tags system as well as traditional fields such as names, phone numbers, and emails. According to the survey data, participants maintain a significant amount of miscellaneous information via tags. To be more specific, the average number of tags among all participants is about 90, the average number of tags per contact is 2.72, and tags account for nearly 30\% of the information in every contact. Moreover, the fact that participants frequently use the tags they store to search also confirms that tags are useful and truly efficient in retaining knowledge of contacts. Concerning the third hypothesis, Graphy supports users in establishing relationships between their contacts. Additionally, by clicking on the relationships users can traverse the relationships network easily. The high number of relationships and relationship navigations affirms the accessibility of these features.

Overall, the user study shows that our new functionalities are helpful and have many of potentials. In fact, the functionalities actually help users solve some existing problems they have in real life. For instance, two participants tell us that almost every contact's first names in their list are made up of names and some other info thus the first names become really long, one participant often has to scroll all the list to find a contact, and to our surprise he does not find that contact by the name (which he has forgotten) but by the event happening when they met. Furthermore, Graphy reveals an innovative behavior from the participants: many of them start creating a \textit{Self} contact in order to connect with other contacts in the relationships network. This special contact turns into a central point which helps the users connect different parts of the network then makes navigations much easier.

On the other hand, the user study still has a few limitations. First, the number of participants is small and our only requirements to a participant are having a smart phone and being familiar with it. Obviously, a larger group of users with more diversity in cultures, educations, professions, and technology familiarities will be much more valuable to the experiment. Second, the duration of the study is fairly short. If the participants have more chance to use the application we may discover more information about their behaviors. Although Graphy is simple and has a shallow learning curve, people still need some time to get used to it. Third, the experiment is conducted in an uncontrolled environment. Despite the fact that most of the users' actions are recorded, we still could not capture the whole interactions between the users and the application. If the study is carried out in a more controlled environment like a lab with cameras or assisted personnels, we will understand the users better. However, this type of controlled study is costly, more limited in duration, and it will compromise users' privacy. Finally, the last limitation of our study is that we did not ask participants to answer any questionnaire. We decided not to require people to fill in a templated questionnaire since we wanted them to only focus on using the application. Nevertheless, with a good questionnaire we can know how people enjoy the system, their difficulties, and we can get more feedbacks and suggestions from them. In our future work, we will tackle these limitations and improve Graphy to capture more users' behaviors.

%\section{Database Synchronization Performance (Will Remove)}
%To evaluate the performance of our synchronization technique, we plan to use Apache Bench and Xamarin to do several load tests on the server such as excuting 1000 requests, processing up to 10 requests concurrently. The results will be measured in milliseconds and compared with other services like Gmail Contacts. 

%We will also benchmark the speed and capacity of the mobile SQLite database. In our Graphy system, the SQLite database runs on the mobile devices and communicates with Xamarin - a cross-platforms development environment. Therefore, the performance of the database can be lower than using a native development environment.

%However, due to the limitation of time and resources, we have not been able to cover all metrics listed above and the user data collected was small. On the collected data, there are some promising results in Custom Tags Weight and Relationship Weight which are shown in Table \ref{tb:experiment}. Although the experiment is still small thus not really comprehensive, it represents a common pattern in users' behaviors: The majority of contacts only contain 2 to 3 fields including the person's name, his/her phone numbers, and his/her organization. Therefore, when the users create custom tags or relationships, the weight of these pieces of information are certainly high. Regarding the performance of Graphy, the database design and the synchronization technique perform very well on the basic daily usage.
%
%\begin{table}[!ht]
%\centering
%\caption{Experiment Results}\label{tb:experiment}
%\begin{tabular}{| l | l | l | l |} \hline
%Criteria & Person 1 & Person 2 & Person 3\\ \hline
%Average CTW & 27.13\% & 22.67\% & 24.84\%\\ \hline
%Average RW & 13.61\% & 19.33\% & 18.32\%\\ \hline
%\end{tabular}
%\end{table}
